\documentclass[b5j,10pt]{tbook}
\usepackage{okumacro}
\usepackage{tate}
\title{処女}
\author{\rensuji{J}山\rensuji{B}作}
\date{}

\begin{document}
\maketitle

恐らくこの文章が僕の処女作になる訳だが。 \\
真面目に何かを書くという行為は、小学生の時の作文以来になる。 \\
\\
人は処女という文字列に、何らかのエピソードを感じ取る事が出来る筈だ。 \\
多くの女性は自身が初めて捧げた相手を思い出すだろう。 \\
幾ばくかの男性は自身に捧げられた女性との行為を思い出すだろう。 \\
\\
 また処女航海や処女作品、他にも様々な処女がある。 \\
\ruby{処}{ところ}により、処女を神聖視する文化もある。 \\
 吸血鬼のモデルとなった人物も、若い処女の血液を求めて殺人を繰り返した。 \\
後世に「処女童貞が吸血されると眷属になる」と転化したのは興味深い。 \\
\\
個人的見解だが、処女に纏わるエピソードには儀式性が付き物だと感じる。 \\
 未通の女性が処女を散らす時 \\
 船が初めて大海原を駆ける時 \\
 人が初めて不死者に出くわす時 \\
\\
そこには儀式があり、何処かへ何かを刻みつける事になるのだ。 \\

\end{document}
